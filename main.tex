\documentclass[12pt]{report}

\usepackage{amssymb,amsthm}
\usepackage{amsmath,color}
%\usepackage{dsfont}
\usepackage{setspace}
\usepackage{graphicx}
\usepackage{multicol}
\usepackage[document]{ragged2e}
\usepackage{fancyhdr}
% \usepackage[]{classicthesis}
\usepackage{hyperref}
\hypersetup{
  colorlinks   = true,    % Colours links instead of ugly boxes
  urlcolor     = blue,    % Colour for external hyperlinks
  linkcolor    = blue,    % Colour of internal links
  citecolor    = red      % Colour of citations
  }
% \usepackage{cref}
  
\def\changemargin#1#2{\list{}{\rightmargin#2\leftmargin#1}\item[]}
\let\endchangemargin=\endlist 

\def\w{\mathtt w}
\def\K{\mathcal K}
\def\N{\mathbb N}
\def\R{\mathcal R}
\def\l{\ell}
\def\S{\mathcal S}
\def\Z{\mathbb Z}
\def\ora{\overrightarrow}


\newtheorem{theorem}{Theorem}[section]
\newtheorem{corollary}{Corollary}[theorem]
\newtheorem{lemma}[theorem]{Lemma}


\begin{document}

\begin{center}
\LARGE{\textbf{Indian Institute of Technology, Delhi}}\\
\vspace{0.8cm}
\large{\textbf{Design Practices in Computer Science}}\\[5pt]
\large{\textbf{COP290}}\\[5pt]

\vspace{0.5cm}

\large{\textbf{Public Distribution Systtem Design}}

\begin{center}
\includegraphics[height=5cm]{iitd.eps}
\end{center}
\vspace{0.2cm}

\textbf{April 18, 2018} \\
\textbf{Department of Computer Science and Engineering} \\
\textbf{Indian Institute of Technology, Delhi}\\


\vspace{1.5cm}

\begin{multicols*}{2}

\begin{flushleft}

\textit{Author :\\ }


\textbf{Udit Jain} \\
(2016CS10327)\\

\end{flushleft}


\columnbreak

\begin{flushleft}

\textit{\\Supervisor :\\ }
\textbf{Prof. Subhashish Bannerjee} \\[5pt]

\end{flushleft}

\end{multicols*}

\end{center}

\newpage

\begin{center}
\Large \bf ABSTRACT
\end{center}
\vspace{0.2in}

I am going to design and present a model for a Public Distribution System for India. In this document , I'll talk about many vital and essential issues that have to be taken into account while designing a Digital Voting system for India. 
\\
\vspace{0.3cm}
The Model will satisfy many properties which make it a good model. Then I will argue thet why is it a very good model to be implemented.


% \vspace{0.5cm}

In this design project, I shall work as observer, developer and algorithm enthusiast to understand the ways and finding different means to approach and tackle the objectives in a more well defined mathematical way. 

\vspace{0.2cm}
% I shall work with full confidence and zeal to achieve the goal or reach to quite an end of the problem so that using our lemmas, proofs and knowledge, someday a perfect model can be implemented using a software by some other Computer Explorer. As a matter of interest, I just wish to argue that these things can be computed by our brain so I do hope to find a solution to this problem using machine learning algorithms. Since, Machine Learning algorithms are more or less based on Mathematical matrices, with the use of computer graphics, I expect to find a start with matrices that I have dealt with further in this report. 

\newpage

\tableofcontents

\newpage

\chapter{Introduction}

The problem of Public Distribution is one that is not limited to India, but a major problem for developing countries. Using the cutting edge technology I hope to present a solution to this problem which will save us plants, workforce and sheer administrative manpower by a once effective and efficient investment by the governing body.
\\
% \cref{sec:baseline}
% {Basic Rules}

\vspace{0.5cm}

The following objectives are aimed to be discussed in this paper:
\begin{itemize}
  \item
  Defining the problem.
  % Support you design decisions ? 
  \item
  Desirable properties 
  % 9 SAK + 1 sankalan + others.
  \item
  Technical and Political Aspects
  % Tradeoffs ? Making strong both  ?
  \item
  Helpful Systems/Algorithms .
  % What are some things models already implemented ?
  \item
  Model  
  % Correctness, Scaling, Implementation, Support for model 
  \item
  Correctness , scaling , feasibility and Implementation
  \item 
  Epilogue
  % Closing the holes, who can we trust ? who will provide/manufacture the harware ? How will it be tested ?
  % How does it help ? What all is improved ? Other applications of this model ?
\end{itemize}

\chapter{Defining the problem}
\section{Introduction}
\subsection{Why do we need a PDS ?}
\section{How do we Currently Distribute ?}
What are the shortfalls of current system, among other description
Who can we trust in the sytstem ?


\chapter{Desirable Properties}
\section{Introduction}
What should be some properties of an PDS ? And how should we decide them ?
\section{Properties}
Now let's formally list the ideally desirable properties of Distribution System.

\chapter{Technical and Political Aspects}
Can a model like this really be implemented ? Who has the power to do this ? Who will conduct this ?
Social aspects ? Proof at every step ?
\section{Key Aspects}

\chapter{Useful Systems}
\section{System in US ?}
Directly packaging the products in tamper proof boxes ? Like LPG cylinders ? 

\chapter{Model}
\section{Software Part}
\section{Hardware Part}
\section{Social Part}
How will people play a role ? How to incentivize/dis-incentivize them ?
\section{Correctness}
How many desirable properties does it satisfy ? How is it feasible for Implementation in the whole country ?
\section{Scalability}
\section{Implementational feasibility}


\chapter{Epilogue}
\section{What does it accomplish ?}
What changes does it bring in the Elections that weren't there previously ?
\section{Applications}
Where else can this system be applied ?
\section{Future Scope}


\chapter{Conclusion}
What are some key learnings throught this excercize ? The power of the system is in it's being fully open source,by the random checks and balances of the people so no middlemen / govt. employess can break it .

% \begin{thebibliography}{9}
% \bibitem{ALM}
% P. Allen, B. Landman and H. Meeks, New Bounds on van der Waerden type numbers for Generalized $3$-term Arithmetic Progressions, {\it arXiv: 1201.3842v2}
% \bibitem{BL}
% S. Burr and S. Loo, On Rado numbers II, unpublished.
% \end{thebibliography}

\end{document}  



